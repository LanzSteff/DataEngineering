%TaskII: Data understanding
\chapter{Data understanding}
The used data for the modeling is in a XLS format and contains 300 instances of software projects from the last 3 years with 9 fields:
\begin{enumerate}
	\item Project ID [nominal]
	\item Country: of the development team [nominal]
	\item FAULT: if it was found a major fault [YES,NO]
	\item Effort: effort in person/months of the original development [numeric]
	\item Module: affected by the changes to fix the errors [nominal]
	\item Component: affected by the changes to fix the errors [C1,...,C7]
	\item BackEnd: indicating if they were also affected by the changes to fix the errors [YES,NO]
	\item FrontEnd: indicating if they were also affected by the changes to fix the errors [YES,NO]
	\item OS: operating system related with the errors [Mac,Windows]
\end{enumerate}

All the entries in the dataset are non-empty. 135 of the entries have a major fault (FAULT: YES), which are 45\% of all entries and 165 entries have no major fault (FAULT: NO), which are 55\% of all entries.
